\section{Background Research}
\subsection{Primary Reaction}
\noindent
This experiment will use the thiocyanatoiron complex reaction as a means to explore the relationship between the equilibrium constant and temperature. The complex formation reaction is described below \citep{buffalostateChemicalEquilibrium}.

\begin{equation*}
    Fe^{3+} + SCN^- \rightleftharpoons FeSCN^{2+}
\end{equation*}

\subsection{Equilibrium Constant}
The equilibrium constant (\(K_c\)) is a measure of the extent to which a chemical reaction proceeds to completion. It is defined as the ratio of the concentrations of the products to the concentrations of the reactants, each raised to the power of their respective stoichiometric coefficients. The equilibrium constant is a crucial parameter in understanding the position of equilibrium in a chemical reaction. If \(K_c < 1\), the equilibrium favors the creation of reactants. If \(K_c > 1\), the equilibrium favors the creation of products. \citep{hammett1935some}

\noindent
\newline
For a reaction:
\begin{equation*}
    aA + bB \rightleftharpoons cC + dD
\end{equation*}

\noindent
The equilibrium constant can be calculated as follows:

\begin{equation}
    K_c=\frac{[C]^c \times [D]^d}{[A]^a \times [B]^b}
\end{equation}

% \begin{figure}[H]
%     \centering
%     \includegraphics[width=130mm,height=\textheight,keepaspectratio]{images/ntc_ptc.png}
%     \caption{Resistance vs Temperature Graph for NTC and PTC Thermistors \citep{amethermptcntc}}
%     \label{fig:ntc_ptc}
% \end{figure}

\subsection{Colorimetry}
Colorimetry is a technique to determine the concentration of a substance in a solution by measuring the absorbance of monochromatic light using a spectrophotometer. It is based on the principle that different substances absorb and transmit light at different wavelengths. By comparing the absorbance of a sample to that of a standard solution, the concentration of the substance in the sample can be determined using the Beer-Lambert Law described below. Colorimetry is widely used in various fields, including environmental analysis, clinical chemistry, and industrial quality control, due to its simplicity, speed, and accuracy \citep{clydesdale1978colorimetry}.

\begin{equation}
    A=\epsilon \; b \; C \label{eq:beer_lambert_law}
\end{equation}

\noindent
Where:
\begin{itemize}[noitemsep,nolistsep]
    \item \(A\) is the absorbance (\(\%\)).
    \item \(\epsilon\) is the molar absorptivity (\(M^{-1} \; cm^{-1}\)).
    \item \(b\) is the path length (\(cm\)).
    \item \(C\) is the molar concentration of the colored sample (\(M\)).
\end{itemize}

\subsection{Importance of Knowing How Temperature Changes the Equilibrium Constant}
Understanding how temperature changes the equilibrium constant of a reaction is crucial for predicting and controlling chemical reactions. The following reaction describes the synthesis of methanol from carbon monoxide and hydrogen gas.

\begin{equation*}
    CO (g) + 2 \; H_2 (g) \rightleftharpoons CH_3OH (g)
\end{equation*}

The reaction is exothermic, meaning it releases heat (\(\Delta H < 0\)). According to Le Chatelier’s principle, for exothermic reactions, the equilibrium constant \(K_c\) decreases with increasing temperature. This means that at higher temperatures, the reaction will shift towards the reactants, producing less methanol. Conversely, at lower temperatures, the reaction will shift towards the products, producing more methanol.

Therefore, in the industrial production of methanol, it would be more beneficial to run the process at a lower temperature to maximize the yield of methanol. However, it's important to note that while lower temperatures favor the production of methanol in terms of thermodynamics, reaction rates are typically slower at lower temperatures. Therefore, a compromise temperature of 250 degrees Celsius is often chosen in industry to balance the thermodynamic favorability with a reasonable reaction rate. Thus, knowledge of how temperature affects the equilibrium constant is essential for optimizing industrial processes and ensuring efficient production of desired products \citep{sciencedirectProductionMethanol}.


% \begin{figure}[H]
%     \centering
%     \includegraphics[width=75mm,height=\textheight,keepaspectratio]{images/conductor_semiconductor_insulator.png}
%     \caption{Resistivity vs Temperature Graph for Conductors, Semiconductors, and Insulators \citep{mandal_2022}. This plot shows the positive TCR of conductors and the negative TCR for semiconductors and insulators.}
%     \label{fig:materials_RT_Plot}
% \end{figure}

\section{Hypothesis}
If the temperature of the solution increases, then the equilibrium constant of the reaction will decrease exponentially. If the temperature of the solution decreases, then the equilibrium constant of the reaction will increase exponentially. 