\section{Conclusion}
Therefore, according to this experiment, there is a very strong correlation (\(R^2=0.9960\)) for a negative exponential relationship between the equilibrium constant and the temperature of a thiocyanatoiron complex reaction.
This matches my hypothesis for this experiment. Furthermore, the individual trial percent uncertainties was far below 5\%, which means that my measurement system is precise. The average percent uncertainties were also far below 5\% which means that the experiment had a precise procedure. Finally, the derived enthalpy and entropy change was well within the accepted range, suggesting fairly accurate results throughout.

\section{Sources for Error}
Error could have been introduced from 4 sources: measurement of temperature using the Vernier Temperature Probe, measurement of transmittance using the spectrophotometer, the indirect heating of the cuvettes using a water bath, and temperature variations during the procedure.

First, the measurement of temperature using the Vernier Temperature Probe contributes to random error since the device uncertainty for the probe is \(\pm\;0.1 \; K\). The random error from this source is negligible in this experiment. Next, the measurement of transmittance using the spectrophotometer also contributes to random error since the device uncertainty for the device is \(\pm\;0.00035\). The random error from this source is also negligible in this experiment. Then, since the cuvettes was indirectly heated through a water bath, the temperature of the water could have been higher than that of the cuvettes leading to systematic error. However, since the heating was put on the lowest setting and the temperature of the cuvettes and the heat bath was normalized by waiting 2-3 minutes, the temperature of the cuvettes should have been relatively close to the temperature of the water bath. Hence, the systematic error from this source is negligible. Finally, as I removed the cuvettes from the hot water bath, wiped off the water on the surface, and placed the sample in the spectrophotometer, the temperature of the small sample likely changed fairly dramatically. This would contribute to a significant systematic error. To account for this, I increased my temperature uncertainty to \(\pm\;1\; K\).

In conclusion, most sources of error were minimal or well-accounted for throughout the experiment.

\section{Strengths}
One strength of this experiment is the use of the spectrophotometer to measure the concentration of the colored \(FeSCN^{2+}\) complex. Other methods to measure the concentration like titration would have taken much longer to complete. Since the spectrophotometer could measure the concentration in a few seconds, minimal temperature was lost from the cuvettes. Furthermore, the ability to take fast measurements enabled me to take a large number of trials and temperature levels. Thus, the use of the spectrophotometer significantly improved the accuracy and precision of this experiment. Another major strength of this experiment was the use of multiple cuvettes in the hot water bath. This allowed me to take all the transmittance measurements for a temperature level at the same time, greatly improving the speed of the experiment. Furthermore, when cuvettes were removed from the water bath, they would decrease in temperature slightly. However, by cycling through the cuvettes, each cuvette has enough time to return to the temperature of the hot water bath. 

\section{Weaknesses}
One weakness of this experiment was the drop in temperature of the cuvettes, while the measurement was taken. Though the measurement time was limited, the drop in temperature was still significant. Another weakness of this experiment was maintaining a consistent temperature for the water bath. Though the hot plate held the temperature relatively constant, it deviated by \(\pm 1^\circ C\), significantly increasing the random error and thus the percent uncertainty in the temperature. These were major experiment design flaws that need to be revised for future trials.

\section{Improvements}
Multiple improvements could be applied to this experiment to address its weaknesses. First, the use of cuvettes made of a more thermally insulating material could reduce the magnitude of the temperature drop. Furthermore, a correction factor could also be developed to adjust the measurements for the impact of temperature drop. Second, the volume of the water bath could be increased. Though it would take a longer period to heat the bath, it would also take a longer period for the bath to cool down. This would greatly diminish the random error from the fluctuating temperature and thus decrease the temperature percent uncertainty. These improvements would greatly improve both the accuracy and precision of this experiment.

\section{Extensions}
A possible extension to this experiment that might be worthy of future study is the relationship between the equilibrium constant and pressure. This experiment explored the relationship between the equilibrium constant and temperature as described by the Van 't Hoff Equation. However, in industrial processes, temperature is modulated along with pressure to achieve ideal operating conditions. Thus, understanding the precise relationship between the equilibrium constant and pressure is significant.
